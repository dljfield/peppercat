\chapter{Implementation}

INTRODUCTION GOES HERE

\section{Graphics}
The graphics renderer was the first piece of the game engine worked on. Having the graphics to provide a view into the game world was important in order to easily see whether things were behaving as expected and proved extremely useful in detecting problems with other parts of the game later on.

\subsection{Isometric Tiles}

\subsubsection{What Is Isometric?}
From the beginning of the project the game specification has listed the world as being presented as \textsc{2d} isometric tiles. While the tiles and the grid structure they sit in is not solely the concern of the renderer, their isometric projection is; the rest of the game is only interested in the position and contents of a grid space, not what it looks like.

When talking about isometric projection in the videogame world it is important to know that it is not truly isometric. True isometric is an ``axonometric projection in which the three coordinate axes appear equally foreshortened and the angles between any two of them are 120 degrees'' [reference wikipedia lol]. What the gaming world calls `isometric' is in fact \textit{dimetric} projection [more reference]---a projection where two axes are the same but the third is not.

The reasons for doing this are essentially one of aesthetics when drawing objects at a very low resolution with visible pixels. At these very coarse resolutions, true isometric projection creates an aesthetically displeasing line, as seen in figure [FIGURE].

However, by changing to a dimetric projection, where the line grows twice as fast horizontally as it does vertically, the line becomes much more consistent and visually appealing. [FIGURE]

The second thing to note is that the term ``\textsc{2d} isometric tile'' used here is a little misleading. \textsc{2d} Cartesian space and isometric space are not the same; rather, \textsc{2d} here refers to the type of graphic used to render the space to the player, that being a simple flat image drawn to look as if were \textsc{3d}.

\subsubsection{Implementing The Projection In Code}
As mentioned in the previous section, the grid that makes up the game world is not, in fact, isometric. The isometric projection exists only in the view of the players. This was done for the simple reason of reducing complexity; the isometric projection has no impact on the underlying structure of the game at all, and so attempting to model the game world in \textsc{3d} is unnecessary.

However, as the game world itself is not modelled in this way, a conversion between the coordinates in the game data and the coordinates on the screen needs to be performed. Luckily, this is incredibly simple:

\begin{lstlisting}[style=js, caption={JavaScript function to turn Cartesian coordinates into game isometric coordinates. Original algorithm from \cite{citeulike:13155325}.}, label=cartesian_to_isometric]
function cartesianToIsometric(cartesian_x, cartesian_y)
{
	var isometric = {};

    isometric.x = cartesian_x - cartesian_y;
    isometric.y = (cartesian_x + cartesian_y) / 2;

    return isometric;
}
\end{lstlisting}

The game world can be thought of as a grid. Each tile on the grid represents a \textsc{2d} Cartesian coordinate, as seen in figure [FIGURE].

% The implementation should look at any issues you encountered as you tried to implement your design. During the work, you might have found that elements of your design were unnecessary or overly complex; perhaps third party libraries were available that simplified some of the functions that you intended to implement. If things were easier in some areas, then how did you adapt your project to take account of your findings?

% It is more likely that things were more complex than you first thought. In particular, were there any problems or difficulties that you found during implementation that you had to address? Did such problems simply delay you or were they more significant?

% You can conclude this section by reviewing the end of the implementation stage against the planned requirements.
