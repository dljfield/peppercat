\chapter{Project Outline Specification}\label{appendix:projectoutline}

%==============================================================================
\section{Project description}
%==============================================================================

\begin{figure}
\centering
	\includegraphics[scale=0.7]{Appendix4/figures/figure-1.png}
	\caption{This image shows a very early build of the game with \textsc{2d} isometric graphics. The blue block represents a player. On the left, the player is obscured by an object; on the right, the player is in front of it.\label{figure_1}}
\end{figure}

An online, multiplayer game based in the browser, utilising \textsc{html5} canvas and JavaScript for the client and Python for the server. It will be played similarly to the table-top roleplaying game \textit{Dungeons \& Dragons}, with a Game Master and several players.

The players will interact with the world in two modes or states. The first is a free-roaming state that operates in real-time. In this state, players will be allowed to interact with the world and the objects and characters within it with relative freedom. The rules that govern them will be loose and mostly describe the interactions of objects with other objects. For example, a magic spell that creates fire would have a heat attribute, and a door made of wood would have a burning point attribute. When they met, if the heat was greater than the burning point, the door would be set on fire.

The other state is the combat state. Players in this state will be locked into a turn-based system where they can only move and perform actions in a limited amount during their turn. Opponents can be creatures or other characters—both player and non-player—and will take their own turns to perform actions. Non-player characters and creatures will be operated by the Game Master. All characters and creatures will have attributes, such as health and mana (for magic), as well as a set of abilities that they can perform. If a character or creature reaches zero health it will have been killed and removed as an active element.

Each player will operate a single character and their view of the world and abilities within it will be defined by that character's location and abilities. If a character is too far away to see another character, then the player will also not see the other character. If a character cannot use magic spells, then the player will not be able to use them.

Player characters who are killed, either in battle or in some other way (perhaps they are killed by falling rocks in the free-roam state) are no longer playable. Players who lose their character may be removed from the game, become a spectator, be given an already existing character previously controlled by the Game Master or be allowed to make a new character.

The Game Master is not a player but rather the controller of the world. They will be able to interact with the world without limitations and be responsible for creating the world via a map editor, guiding the players around it, operating non-player characters and creatures in the world and set up events for the players. The Game Master will even be able to override the normal rules of the world. In the example of the fire spell and the door, the Game Master will be able to say that the door is not set on fire, even if it otherwise would have been.

The world will be presented to users using tile-based, \textsc{2d} isometric graphics (as seen in Figure \ref{figure_1}). It will consist of a planar terrain with objects, items and characters on top of it. Movement in the world will be done in 8 directions: up, down, left, right and diagonal. Characters and creatures will move from tile to tile, with only one able to be in a tile at any given time. However, each tile can hold many items (such as weapons, money, clothes). Objects will be varied, with those such as walls and pillars taking up a tile by themselves but objects such as chairs or chests of treasure that can be interacted with by characters may coexist in a tile with characters.

The players and Game Master will be able to communicate throught a textual chat system. In its most basic form, this will be a global chat that all users in the game can see. However, the ability to restrict chat to a local context or to an individual user would be nice.

Extra features that would enhance the project but are not mandatory include: voice chat system; multi-levelled maps with varying heights; random map generator; random creature/character generator; visual representation for character items (such as weapons, clothes, etc); and the ability for users to upload their own artwork for use in their games.

%==============================================================================
\section{Proposed tasks}
%==============================================================================
My proposed tasks to achieve a basic version of the project are as follows:

\begin{itemize}
	\item{Create basic client-side graphics engine, allowing a map to be rendered and a player to move around.}
	\item{Create basic multiplayer functionality, setting up the server and allowing multiple players to exist in the same map and move around within it.}
	\item{Add Game Master, who can select different characters to control.}
	\item{Add items and attributes to characters, allowing them to carry things and setting things up for the next task.}
	\item{Add combat state, allowing users to do more than walk around the world.}
	\item{Add attribute interactions in the free-roam state, allowing for the fire spell and wooden door interaction.}
	\item{Add map editor for the Game Master.}
\end{itemize}

%==============================================================================
\section{Project deliverables}
%==============================================================================
\begin{description}
	\item{\textbf{Game Client,} final `production' version.} This is what the users of the game will interact with, via a browser, allowing for both regular players and a Game Master, who is given the ability to create maps for the game.

	\item{\textbf{Game Server,} final `production' version.} This will be responsible for syncing the game between all the players and providing authoritative state to clients to help prevent cheating.

	\item{\textbf{Documentation.}} Basic guides for users that instruct them on how to operate the game from the Game Master and player perspectives.

	\item{\textbf{Final Report.}} Report detailing the system; the process of the system's creation from beginning to end; differences between the proposal and final system and explanations for those differences; full bibliography.
\end{description}