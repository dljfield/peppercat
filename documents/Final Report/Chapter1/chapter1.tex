\chapter{Background \& Objectives}

Games are interesting projects to take on; they have a history of being difficult to make and pushing technology to its limits. There are a great many systems and features that can be included in them---AI, physics, graphics, audio, UI, multiplayer and more---and a great many ways to implement each, from simple to very complicated depending on the needs of the project.

Games also have a history of aiming for too many of these features in too little time. In most cases, every one of the features thought up would enhance the final product in some way, from a significant improvement that changes the way the game is played to a minor enhancement that makes things just a little more pleasant for the user. Many of these features may be considered mandatory for the game to be worth making at all. For example, a single-player chess game would probably not be very good if there were no AI to play against.

It is not just players who require features, however; developers of games need tools to implement the game design and a good engine to hang the design off of. A lot of game projects build custom tools that let developers and designers implement things quickly. Many game engines even provide methods for scripting and modding them after their release to players.

It is clear that, given the sheer enormity of the possible things that can be put into any one game, there is not enough time in this project to implement even half of them without a great deal of previous experience and skill. Every one of the major systems mentioned can be extremely complicated, requiring a lot of research, time and effort to make them work.

This chapter will discuss what the project is, why it was worth taking on and how a minimal system was devised that would satisfy enough of the game design requirements to be playable but also be implementable in the time given.

\section{Background}

\subsection{The Project}
Before discussing the decisions made about what was doable and why it was interesting, it's useful to know what the project actually is. A full account of the original, idealised, game design can be found in Appendix \ref{appendix:designspec}.

The name of the project---\textit{Browser-based Online Multiplayer Roleplaying Game}---gives a relatively good hint as to what the game is. ``Browser-based'' and ``multiplayer'' are fairly self-evident in meaning: multiple people play together in a game hosted in the browser. ``Roleplaying game'' is more ambiguous. In this case, it refers to a game in the style of the classic tabletop roleplaying game \textit{Dungeons \& Dragons.}

In the context of the project that meant the following things: Firstly, there needed to be two types of players---regular player and Game Master. A regular player plays the game as a character inhabiting the world they happen to be in. For example, they may be a dwarf in a fantasy kingdom, or a space marine on a futuristic space station.

The Game Master is a player responsible for building the world, telling the story and controlling characters that aren't controlled by the players (known as Non-Player Characters or NPCs). Traditionally, the Game Master would also be responsible for enforcing the rules of the world. However, in this project the game was to be responsible for that instead. The Game Master could, however, override or change the rules if he or she wished to do so.

Combat is the most obvious area where the game enforcing the rules comes into effect, as tradtionally combat is the strictest area of tabletop gameplay. Combat is turn-based, with players put onto a grid and given limits on the distance they can move and number of actions they can perform in each turn. When they attempt to do something---such as attack another character or creature or escape from a trap---they have to roll dice, the result of which decides whether they were successful or not, and often how well they succeeded or failed. As an example of the last part, a player failing to attack a creature with their sword could simply miss, or they could throw the sword away accidentally, depending on how badly they failed.

The project design specification called for interactions outside of combat too. For example, a player might be faced with a locked door. To get through a player could attempt to use a key they found. Alternatively, they could attempt to bash the door open with an item, such as an axe, or even their bare hands.

Given that the majority of the game is players---be they regular or Game Master---interacting with each other it was important to allow players to communciate effectively. To this end, a chat system was specified.

How the world is presented to players is important too. Given the style of game a simple textual system may well work. However, in this case graphics were asked for. More specifically, the graphics needed to be 2D isometric tiles.

Finally, there were some technical specifications as well. Firstly, the game was to be run in a browser. Secondly, the game is multiplayer and needed a server, for which Python was chosen as the language with the goal of gaining experience in it.

\subsection{Why make the game?}
It is important to answer why the project was worth doing, particularly as it was a student-suggested one.

The first answer to this is that games are interesting in general. Most obviously, the final product of a game is (hopefully) something fun to play with appeal to a wide range of people. More relevant to the context of a project, however, is that games are interesting from a software perspective.

Games are made up of a lot of different parts, each one potentially being difficult to implement by itself. In this game, the most challenging individual parts are graphics and multiplayer. More important than just the individual parts, however, is making sure they fit together and work properly. In most cases the game needs to share data between these different places---game logic needs to know what an object is doing so it can perform game functions on it; the renderer needs to know what the object is doing so that it can be drawn to the screen correctly; the networking part needs to know what the object is doing so that it can forward on any relevant information to the server.

The game offers a lot of extendability. Given more time, more features can be added. Each feature, and fitting it together, offers a lot of potential for learning as well. For example, an extra feature could be AI, which is an interesting area in itself that offers a lot of opportunity to learn something new.

\subsection{Other Games}
It is useful to take note of other games in the space that this project inhabits, both to see whether the project is worth doing and whether there is anything to be learned from the work of others.




\section{Analysis}
With the knowledge that the scope of games can expand to encompass a ridiculous area, and that everything within that scope is likely to be time consuming to implement, it is important to do some research into each feature that one would like to include.

Taking into account the problem and what you learned from the background work, what was your analysis of the problem? How did your analysis help to decompose the problem into the main tasks that you would undertake? Were there alternative approaches? Why did you choose one approach compared to the alternatives?

There should be a clear statement of the objectives of the work, which you will evaluate at the end of the work.

In most cases, the agreed objectives or requirements will be the result of a compromise between what would ideally have been produced and what was felt to be possible in the time available. A discussion of the process of arriving at the final list is usually appropriate.


\section{Process}
You need to describe briefly the life cycle model or research method that you used. You do not need to write about all of the different process models that you are aware of. Focus on the process model that you have used. It is possible that you needed to adapt an existing process model to suit your project; clearly identify what you used and how you adapted it for your needs.

